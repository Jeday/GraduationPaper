% В этом файле следует писать текст работы, разбивая его на
% разделы (section), подразделы (subsection) и, если нужно,
% главы (chapter).

% Предварительно следует указать необходимую информацию
% в файле SETUP.tex

%% В этот файл не предполагается вносить изменения

% В этом файле следует указать информацию о себе
% и выполняемой работе.

\documentclass [fontsize=14pt, paper=a4, pagesize, DIV=calc]%
{scrartcl}
% ВНИМАНИЕ! Для использования глав поменять
% scrartcl на scrreprt

% Здесь ничего не менять
\usepackage [T2A] {fontenc}   % Кириллица в PDF файле
\usepackage [utf8] {inputenc} % Кодировка текста: utf-8
\usepackage [russian] {babel} % Переносы, лигатуры

%%%%%%%%%%%%%%%%%%%%%%%%%%%%%%%%%%%%%%%%%%%%%%%%%%%%%%%%%%%%%%%%%%%%%%%%
% Создание макроса управления элементами, специфичными
% для вида работы (курс., бак., маг.)
% Здесь ничего не менять:
\usepackage{ifthen}
\newcounter{worktype}
\newcommand{\typeOfWork}[1]
{
	\setcounter{worktype}{#1}
}

% ВНИМАНИЕ!
% Укажите тип работы: 0 - курсовая, 1 - бак., 2 - маг.,
% 3 - бакалаврская с главами.
\typeOfWork{1}
% Считается, что курсовая и бак. бьются на разделы (section) и
% подразделы (subsection), а маг. — на главы (chapter), разделы и
%  подразделы. Если хочется,
% чтобы бак. была с главами (например, если она большая),
% надо выбрать опцию 3.

% Если при выборе 2 или 3 вы забудете поменять класс
% документа на scrreprt (см. выше, в самом начале),
% то получите ошибку:
% ./aux/appearance.tex:52: Package scrbase Error: unknown option ` chapterprefix=

%%%%%%%%%%%%%%%%%%%%%%%%%%%%%%%%%%%%%%%%%%%%%%%%%%%%%%%%%%%%%%%%%%%%%%%%
% Информация об авторе и работе для титульной страницы

\usepackage {titling}

% Имя автора в именительном падеже (для маг.)
\newcommand {\me}{%
Е.\,А.~Тактаров%
}

% Имя автора в родительном падеже (для курсовой и бак.)
\newcommand {\byme}{%
Е.\,А.~Тактарова%
}

% Научный руководитель
\newcommand{\supervisor}%
{к.ф.-м.н., доцент Е. М. Андреева}

% идентифицируем пол (только для курсовой и бак.)
\newcommand{\bystudent}{
Студента %Студентки % Для курсовой: с большой буквы
}

% Год публикации
\date{2019}

% Название работы
\title{Разработка системы проведения опросов аудитории\\ во время публичных выступлений}

% Кафедра
%

\newcommand {\direction} {%
Направление подготовки\\02.\ifthenelse{\value{worktype} = 2}{04}{03}.02 ---
Фундаментальная информатика\\и информационные технологии%
}

%%%%%%%%%%%%%%%%%%%%%%%%%%%%%%%%%%%%%%%%%%%%%%%%%%%%%%%%%%%%%%%%%%%%%%%%
% Другие настраиваемые элементы текста

% Листинги с исходным кодом программ: укажите язык программирования
\usepackage{listings}
\lstset{
    language=[ISO]C++,%  Язык указать здесь
    basicstyle=\small\ttfamily,
    breaklines=true,%
    showstringspaces=false%
    inputencoding=utf8x%
}
% полный список языков, поддерживаемых данным пакетом, есть,
% например, здесь (стр. 13):
% ftp://ftp.tex.ac.uk/tex-archive/macros/latex/contrib/listings/listings.pdf

\usepackage{color}
\definecolor{lightgray}{rgb}{.9,.9,.9}
\definecolor{darkgray}{rgb}{.4,.4,.4}
\definecolor{purple}{rgb}{0.65, 0.12, 0.82}
\lstdefinelanguage{JavaScript}{
	keywords={break, case, catch, continue, debugger, default, delete, do, else, false, finally, for, function, if, in, instanceof, new, null, return, switch, this, throw, true, try, typeof, var, void, while, with},
	morecomment=[l]{//},
	morecomment=[s]{/*}{*/},
	morestring=[b]',
	morestring=[b]",
	ndkeywords={class, export, boolean, throw, implements, import, this,require},
	keywordstyle=\color{blue}\bfseries,
	ndkeywordstyle=\color{cyan}\bfseries,
	identifierstyle=\color{black},
	commentstyle=\color{purple}\ttfamily,
	stringstyle=\color{red}\ttfamily,
	sensitive=true
}


\usepackage{xcolor}

\colorlet{punct}{red!60!black}
\definecolor{background}{HTML}{EEEEEE}
\definecolor{delim}{RGB}{20,105,176}
\colorlet{numb}{magenta!60!black}

\lstdefinelanguage{json}{
	basicstyle=\normalfont\ttfamily,
	numbers=left,
	numberstyle=\scriptsize,
	stepnumber=1,
	numbersep=8pt,
	showstringspaces=false,
	breaklines=true,
	frame=lines,
	backgroundcolor=\color{background},
	literate=
	*{0}{{{\color{numb}0}}}{1}
	{1}{{{\color{numb}1}}}{1}
	{2}{{{\color{numb}2}}}{1}
	{3}{{{\color{numb}3}}}{1}
	{4}{{{\color{numb}4}}}{1}
	{5}{{{\color{numb}5}}}{1}
	{6}{{{\color{numb}6}}}{1}
	{7}{{{\color{numb}7}}}{1}
	{8}{{{\color{numb}8}}}{1}
	{9}{{{\color{numb}9}}}{1}
	{:}{{{\color{punct}{:}}}}{1}
	{,}{{{\color{punct}{,}}}}{1}
	{\{}{{{\color{delim}{\{}}}}{1}
	{\}}{{{\color{delim}{\}}}}}{1}
	{[}{{{\color{delim}{[}}}}{1}
	{]}{{{\color{delim}{]}}}}{1},
}




% Нумерация списков: можно при необходимести
% изменять вид нумерации (например, добавлять правую скобку).
% По умолчанию буду списки вида:
% 1.
% 2.
% Изменять вид нумерации можно в начале нумерации:
% \begin{enumerate}[1)] (В квадратных скобках указан желаемый вид)
\usepackage[shortlabels]{enumitem}
                    \setlist[enumerate, 1]{1.}

% Гиперссылки: настройте внешний вид ссылок
\usepackage%
[pdftex,unicode,pdfborder={0 0 0},draft=false,%backref=page,
    hidelinks, % убрать, если хочется видеть ссылки: это
               % удобно в PDF файле, но не должно появиться на печати
    bookmarks=true,bookmarksnumbered=false,bookmarksopen=false]%
{hyperref}


\usepackage {amsmath}      % Больше математики
\usepackage {amssymb}
\usepackage {textcase}     % Преобразование к верхнему регистру
\usepackage {indentfirst}  % Красная строка первого абзаца в разделе

\usepackage {fancyvrb}     % Листинги: определяем своё окружение Verb
\DefineVerbatimEnvironment% с уменьшенным шрифтом
	{Verb}{Verbatim}
	{fontsize=\small}

% Вставка рисунков
\usepackage {graphicx}

% Общее оформление
% ----------------------------------------------------------------
% Настройка внешнего вида

%%% Шрифты

% если закомментировать всё — консервативная гарнитура Computer Modern
\usepackage{paratype} % профессиональные свободные шрифты
%\usepackage {droid}  % неплохие свободные шрифты от Google
%\usepackage{mathptmx}
%\usepackage {mmasym}
%\usepackage {psfonts}
%\usepackage{lmodern}
%var1: lh additions for bold concrete fonts
%\usepackage{lh-t2axccr}
%var2: the package below could be covered with fd-files
%\usepackage{lh-t2accr}
%\usepackage {pscyr}

% Геометрия текста

\usepackage{setspace}       % Межстрочный интервал
\onehalfspacing

\newlength\MyIndent
\setlength\MyIndent{1.25cm}
\setlength{\parindent}{\MyIndent} % Абзацный отступ
\frenchspacing            % Отключение лишних отступов после точек
\KOMAoptions{%
    DIV=calc,         % Пересчёт геометрии
    numbers=endperiod % точки после номеров разделов
}

                            % Консервативный вариант:
%\usepackage                % ручное задание геометрии
%[%                         % (не рекомендуется в проф. типографии)
%  margin = 2.5cm,
  %includefoot,
  %footskip = 1cm
%] %
%  {geometry}

%%% Заголовки


\ifthenelse{{\value{worktype} > 1}}{%
  \KOMAoptions{%
      headings=normal,   % размеры заголовков поменьше стандартных
      chapterprefix=true,% Печатать слово Глава
      appendixprefix=true% Печатать слово Приложение
  }
}{% Печатать слово Приложение даже если нет глав
  \newcommand*{\appendixmore}{%
    \renewcommand*{\sectionformat}{%
    \appendixname~\thesection\autodot\enskip}
    \renewcommand*{\sectionmarkformat}{%
      \appendixname~\thesection\autodot\enskip}
  }
}

% шрифт для оформления глав и названия содержания
\newcommand{\SuperFont}{\Large\sffamily\bfseries}

% Заголовок главы
\ifthenelse{\value{worktype} > 1}{%
\renewcommand{\SuperFont}{\Large\normalfont\sffamily}
\newcommand{\CentSuperFont}{\centering\SuperFont}
\usepackage{fncychap}
\ChNameVar{\SuperFont}
\ChNumVar{\CentSuperFont}
\ChTitleVar{\CentSuperFont}
\ChNameUpperCase
\ChTitleUpperCase
}

% Заголовок (под)раздела с абзацного отступа
\addtokomafont{sectioning}{\hspace{\MyIndent}}

\renewcommand*{\captionformat}{~---~}
\renewcommand*{\figureformat}{Рисунок~\thefigure}

% Плавающие листинги
\usepackage{float}
\floatstyle{ruled}
\floatname{ListingEnv}{Листинг}
\newfloat{ListingEnv}{htbp}{lol}[section]

% точка после номера листинга
\makeatletter
\renewcommand\floatc@ruled[2]{{\@fs@cfont #1.} #2\par}
\makeatother


%%% Оглавление
\usepackage{tocloft}

% шрифт и положение заголовка
\ifthenelse{\value{worktype} > 1}{%
\renewcommand{\cfttoctitlefont}{\hfil\SuperFont\MakeUppercase}
}{
\renewcommand{\cfttoctitlefont}{\hfil\SuperFont}
}

% слово Глава
\usepackage{calc}
\ifthenelse{\value{worktype} > 1}{%
\renewcommand{\cftchappresnum}{Глава }
\addtolength{\cftchapnumwidth}{\widthof{Глава }}
}

% Очищаем оформление названий старших элементов в оглавлении
\ifthenelse{\value{worktype} > 1}{%
\renewcommand{\cftchapfont}{}
\renewcommand{\cftchappagefont}{}
}{
\renewcommand{\cftsecfont}{}
\renewcommand{\cftsecpagefont}{}
}

% Точки после верхних элементов оглавления
\renewcommand{\cftsecdotsep}{\cftdotsep}
%\newcommand{\cftchapdotsep}{\cftdotsep}

\ifthenelse{\value{worktype} > 1}{%
    \renewcommand{\cftchapaftersnum}{.}
}{}
\renewcommand{\cftsecaftersnum}{.}
\renewcommand{\cftsubsecaftersnum}{.}
\renewcommand{\cftsubsubsecaftersnum}{.}

%%% Списки (enumitem)

\usepackage {enumitem}      % Списки с настройкой отступов
\setlist %
{ %
  leftmargin = \parindent, itemsep=.5ex, topsep=.4ex
} %

% По ГОСТу нумерация должны быть буквами: а, б...
%\makeatletter
%    \AddEnumerateCounter{\asbuk}{\@asbuk}{м)}
%\makeatother
%\renewcommand{\labelenumi}{\asbuk{enumi})}
%\renewcommand{\labelenumii}{\arabic{enumii})}

%%% Таблицы: выбрать более подходящие

\usepackage{booktabs} % считаются наиболее профессионально выполненными
%\usepackage{ltablex}
%\newcolumntype {L} {>{---}l}

%%% Библиография

\usepackage{csquotes}        % Оформление списка литературы
\usepackage[
  backend=biber,
  hyperref=auto,
  sorting=none, % сортировка в порядке встречаемости ссылок
  language=auto,
  citestyle=gost-numeric,
  bibstyle=gost-numeric
]{biblatex}
\addbibresource{biblio.bib} % Файл с лит.источниками

% Настройка величины отступа в списке
\ifthenelse{\value{worktype} < 2}{%
\defbibenvironment{bibliography}
  {\list
     {\printtext[labelnumberwidth]{%
    \printfield{prefixnumber}%
    \printfield{labelnumber}}}
     {\setlength{\labelwidth}{\labelnumberwidth}%
      \setlength{\leftmargin}{\labelwidth}%
      \setlength{\labelsep}{\dimexpr\MyIndent-\labelwidth\relax}% <----- default is \biblabelsep
      \addtolength{\leftmargin}{\labelsep}%
      \setlength{\itemsep}{\bibitemsep}%
      \setlength{\parsep}{\bibparsep}}%
      \renewcommand*{\makelabel}[1]{\hss##1}}
  {\endlist}
  {\item}
}{}

% ----------------------------------------------------------------
% Настройка переносов и разрывов страниц

\binoppenalty = 10000      % Запрет переносов строк в формулах
\relpenalty = 10000        %

\sloppy                    % Не выходить за границы бокса
%\tolerance = 400          % или более точно
\clubpenalty = 10000       % Запрет разрывов страниц после первой
\widowpenalty = 10000      % и перед предпоследней строкой абзаца

% ----------------------------


% Стили для окружений типа Определение, Теорема...
% Оформление теорем (ntheorem)

\usepackage [thmmarks, amsmath] {ntheorem}
\theorempreskipamount 0.6cm

\theoremstyle {plain} %
\theoremheaderfont {\normalfont \bfseries} %
\theorembodyfont {\slshape} %
\theoremsymbol {\ensuremath {_\Box}} %
\theoremseparator {:} %
\newtheorem {mystatement} {Утверждение} [section] %
\newtheorem {mylemma} {Лемма} [section] %
\newtheorem {mycorollary} {Следствие} [section] %

\theoremstyle {nonumberplain} %
\theoremseparator {.} %
\theoremsymbol {\ensuremath {_\diamondsuit}} %
\newtheorem {mydefinition} {Определение} %

\theoremstyle {plain} %
\theoremheaderfont {\normalfont \bfseries} 
\theorembodyfont {\normalfont} 
%\theoremsymbol {\ensuremath {_\Box}} %
\theoremseparator {.} %
\newtheorem {mytask} {Задача} [section]%
\renewcommand{\themytask}{\arabic{mytask}}

\theoremheaderfont {\scshape} %
\theorembodyfont {\upshape} %
\theoremstyle {nonumberplain} %
\theoremseparator {} %
\theoremsymbol {\rule {1ex} {1ex}} %
\newtheorem {myproof} {Доказательство} %

\theorembodyfont {\upshape} %
%\theoremindent 0.5cm
\theoremstyle {nonumberbreak} \theoremseparator {\\} %
\theoremsymbol {\ensuremath {\ast}} %
\newtheorem {myexample} {Пример} %
\newtheorem {myexamples} {Примеры} %

\theoremheaderfont {\itshape} %
\theorembodyfont {\upshape} %
\theoremstyle {nonumberplain} %
\theoremseparator {:} %
\theoremsymbol {\ensuremath {_\triangle}} %
\newtheorem {myremark} {Замечание} %
\theoremstyle {nonumberbreak} %
\newtheorem {myremarks} {Замечания} %


% Титульный лист
% Макросы настройки титульной страницы
% В этот файл не предполагается вносить изменения

%\usepackage {showframe}

% Вертикальные отступы на титульной странице
\newcommand{\vgap}{\vspace{16pt}}

% Помещение города и даты в нижний колонтитул
\usepackage{scrlayer}
\DeclareNewLayer[
  foot,
  foreground,
  contents={%
    \raisebox{\dp\strutbox}[\layerheight][0pt]{%
      \parbox[b]{\layerwidth}{\centering Ростов-на-Дону\\ \thedate%
       \\\mbox{}
       }}%
  }
]{titlepage.foot.fg}
\DeclareNewPageStyleByLayers{titlepage}{titlepage.foot.fg}


\AtBeginDocument %
{ %
  %
  \begin{titlepage}
  %
    \thispagestyle{titlepage}

    {\centering
    %
    \MakeTextUppercase {МИНИСТЕРСТВО ОБРАЗОВАНИЯ И НАУКИ РФ}

    \vgap

    Федеральное государственное автономное образовательное\\
    учреждение высшего образования\\
    \MakeTextUppercase {Южный федеральный университет}

    \vgap

	Институт математики, механики и компьютерных наук
    имени~И.\,И.\,Воровича

    \vgap

    \direction

    \vspace* {\fill}

    \ifthenelse{\value{worktype} = 2}{%
    \me

    \vgap}{}

    {\usefont{T2A}{PTSansCaption-TLF}{m}{n}
    \MakeTextUppercase{\thetitle}}

    \ifthenelse{\value{worktype} = 2}{%
     \vgap

    Магистерская диссертация}{}
    \ifthenelse{\value{worktype} = 0}{
     \vgap

    Курсовая работа
    }{}%
    \ifthenelse{\value{worktype} = 1 \OR \value{worktype} = 3}{
     \vgap

    Выпускная квалификационная работа\\
    на степень бакалавра
    }{}%

    \vspace {\fill}

    \begin{flushright}
    \ifthenelse{\value{worktype} = 0 \OR 
                \value{worktype} = 1 \OR
                \value{worktype} = 3}{
      \bystudent \ifthenelse{\value{worktype} = 0}{3}{4}\ курса\\
      \byme
    }{}

    \vgap

    Научный руководитель:\\
    \supervisor\\
    \ifthenelse{\value{worktype} = 2}{%
    Рецензент:\\
    ученая степень, ученое звание, должность
    И. О. Фамилия
    }{}
	\end{flushright}
\ifthenelse{\value{worktype} = 0}{
\vspace{\fill}
        \begin{flushleft}
          \begin{tabular}{cc}
            \underline{\hspace{4cm}}&\underline{\hspace{5cm}}\\
            {\small оценка (рейтинг)} & {\small  подпись руководителя}\\
          \end{tabular}
          \\[1cm]
        \end{flushleft}
}{}
\ifthenelse{\value{worktype} = 1 \OR \value{worktype} = 3}{
\vspace{\fill}
        \begin{flushleft}
Допущено к защите:\\руководитель направления ФИИТ
\underline{\hspace{4cm}}
В.\,С.\,Пилиди
        \end{flushleft}
}{}


  	\vspace {\fill}
  %Ростов-на-Дону

    %\thedate

  }\end{titlepage}
  %
  %
  \tableofcontents
  %
  \clearpage
} %


% Команды для использования в тексте работы


% макросы для начала введения и заключения
\newcommand{\Intro}{\addsec{Введение}}
\ifthenelse{\value{worktype} > 1}{%
    \renewcommand{\Intro}{\addchap{Введение}}%
}

\newcommand{\Conc}{\addsec{Заключение}}
\ifthenelse{\value{worktype} > 1}{%
    \renewcommand{\Conc}{\addchap{Заключение}}%
}

% Правильные значки для нестрогих неравенств и пустого множества
\renewcommand {\le} {\leqslant}
\renewcommand {\ge} {\geqslant}
\renewcommand {\emptyset} {\varnothing}

% N ажурное: натуральные числа
\newcommand {\N} {\ensuremath{\mathbb N}}

% значок С++ — используйте команду \cpp
\newcommand{\cpp}{%
C\nolinebreak\hspace{-.05em}%
\raisebox{.2ex}{+}\nolinebreak\hspace{-.10em}%
\raisebox{.2ex}{+}%
}

% Неразрывный дефис, который допускает перенос внутри слов,
% типа жёлто-синий: нужно писать жёлто"/синий.
\makeatletter
    \defineshorthand[russian]{"/}{\mbox{-}\bbl@allowhyphens}
\makeatother


\endinput

% Конец файла



\NewBibliographyString{langjapanese}
\NewBibliographyString{fromjapanese}

\begin{document}

\Intro
Технологии проведения публичных выступлений и презентаций затрагивают навыки ораторства и внешний вид, дизайн медиа-сопровождения. Методы взаимодействия с аудиторией традиционно включают в первую составляющую. Выступающий, желающий взаимодействовать со слушающими, должен уже обладать определенным опытом в работе с ними и ограничен устными средствами. Крайне редко возможно почти полностью вовлечь аудиторию в выступление, ведь лишь немногие слушатели готовы, например, задать вопрос или ответить выступающему.


Распространение телефонов и мобильного доступа в интернет, позволяет использовать эти устройства как средства взаимодействия с аудиторией. Проекты, использующие эту идею, реализовывались неоднократно, но ни один из них не закрепился как широко используемый в публичных выступлениях. В первую очередь, идея взаимодействия с публикой через телефоны реализовывалась под конкретные единичные выступления. Последующие реализации, хотя и обладают обширным функционалом, в виде  опросов, голосований и чатов, представлют собой отдельные веб-сервисы, направленные на монетизацию с пользователей. Все проекты закрыты проприетарными лицензиями и требуют от пользователей загрузки презентации на сторонний сервер.

Данная работа посвящена разработке проекта портативного веб-сервиса под свободной лицензией, который позволит проводить опросы аудитории во время публичных выступлений без привлечения сторонних сервисов. Свободная лицензия позволит любому человеку изменять и расширять возможности сервиса под свои нужды.

Задача по созданию такого проекта включает разработку как веб-интерфейса пользователя (фронтенд), так и  внутренней логики сервиса (бэкенд), которые в совокупности обеспечат динамичное отображение результатов опросов.
   
\section{Исследование предметной области}
\subsection{Обзор существующих решений}
Как и упоминалось ранее, для опросов аудитории уже существует немалое число инструментов, однако в основой массе это закрытые решения в виде веб-сервисов:
\begin{itemize}
	\item polleverywhere.com
	\item directpoll.com
	\item sli.do
	\item ficus.io
\end{itemize}
На этих сайтах и других подобных можно бесплатно один раз провести опрос или даже презентацию, но повторные показы и дополнительные функции ограничены для пользователей, не оплативших услуги сайтов. Более того, даже оплативший пользователь ограничен средствами и функциями сайта и не может модифицировать или изменить инструмент под свои нужды и цели.

Также стоит упомянуть об инструментах опросов, не использующих только Интернет~\autocite{ombea}. Такие решения применяются в университетах США~\autocite{nea} и отличаются низкой способностью к масштабированию и высокой ценой как системы, так и индивидуальных приборов голосования.

\subsection{Обзор инструментов разработки}
При создании веб-сервиса самую важную роль занимает разработка серверной части. Так как веб-сайт должен динамически взаимодействовать с сервером, то архаичная связка из веб-сервера и CGI приложения очевидно не подойдет. Для решения данной задачи необходимо выбрать один из множества современных веб-фреймворков~\autocite{wiki}, как основу для проекта. Отметим основные необходимые для задачи черты фреймворков:
 \begin{enumerate}
 	\item легковесность
 	\item инкапсуляция веб-сервера
 	\item наличие актуального функционала(\textbf{JSON}, \textbf{AJAX}, \textbf{WebSocket})
 \end{enumerate}
Рассмотрим несколько популярных фреймворков:
 \begin{itemize}
 	\item \textbf{Django} --- фреймворк на языке Python. Хотя на нем можно реализовать необходимый нам функционал, но его врядли можно назвать легковесным. Django в первую очередь предназначен для создания больших многостраничных сайтов и сервисов, которые будет длительное время поддерживать команда разработчиков и администраторов. Наличие бесполезного для задачи функционалла негативно сказывается на времени освоения и разработки~\autocite{django}.  
 	\item \textbf{Ruby on Rails} --- фреймворк на языке Ruby. Основными минусами Ruby on Rails являются проксирование через отдельный веб-сервер и общая сложность освоения как фреймворка, так и самого языка. Стоит также отметить, что этот фреймворк сильно опирается на архитектуру модель-представление-контроллер, реализация которой усложняет задачу для небольшого приложения~\autocite{ruby}.  
 	\item \textbf{Express} --- фреймворк на языке JavaScript, запускаемый на платформе Node.js. Express инкапсулирует веб-сервер,представляя только абстракцию в виде объектов HTTP запроса и ответа, а необходимый для приложения функционал, например WebSocket и шаблонизация, добавляются через совместимые модули Node.js. Так же фреймворк не затрагивает клиентскую часть веб-приложения~\autocite{express}. Express своевременно обновляется, имеет обширную документацию и является популярным выбором среди разработчиков из-за своей простоты и понятности. Основным опасением является производительность однопоточной архитектуры Node.js, однако для небольших и средних приложений Node.js и Express показывают удовлетворительные результаты~\autocite{Kai14}.
 \end{itemize}
Таким образом, для решения рассматриваемой задачи фреймворк Express подходит идеально. Так как он не затрагивает клиентскую часть сервиса, то для нее можно использовать любой легковесный JavaScript-фреймворк для создания пользовательского интерфейса.

Также важно выбрать технологию обмена данных между сервером и клиентом. Так как в нашем случае сервер должен отправлять данные, даже когда клиент не запрашивает их явно, то данное веб-приложение соотвествует модели \textbf{COMET}~\autocite{Krill07}. Рассмотрим некоторые технологии реализации \textbf{COMET} в современных веб-приложениях:

\begin{description}
	\item[Спрятанный iframe] HTML-элемент \textbf{iframe}, который подгружает <<бесконечную>> подстраницу с сервера, состоящую из элементов \textbf{script} с данными. Такой метод требует особой модификации веб-сервера, но поддерживается во всех браузерах. Достаточно сложен в реализации как на сервере, так и на клиенте. 
	\item[подгружаемый XMLHttpRequest] аналогично с предыдущим методом, но вместо страницы загружается AJAX-запрос.
	\item[XMLHttpRequest long polling] в этой технологии браузер отправляет AJAX-запрос, но ответ получает, только когда серверу нужно отправить данные клиенту. 
	\item[WebSocket] отдельный протокол двустороннего соединения поверх TCP. Его реализация есть во всех современных браузерах и серверах, в том числе и для Node.js. По сравнению с другими технологиями WebSocket отличается производительностью, скоростью передачи данных и удобством в работе.
\end{description}

Исходя из этого, для данной задачи лучше всего использовать технологию WebSocket.

Реализация современных веб-сайтов без использования фреймворков для создания интерфейса и дизайна не является актуальной задачей. Вручную описывать манипуляции с иерархической структурой HTML-страницы на чистом JavaScript крайне сложно даже для данного небольшого проекта с динамическим содержанием. Стоит отметить, что разнообразие размеров экранов мобильных устройств и персональных компьютеров также осложняет ручную верстку страницы.

Существует огромное количество фронтенд-фреймворков, вряд ли среди них можно выбрать абсолютно лучший, поэтому выбор фреймворка для маленького или среднего проектов это личное предпочтение. Я буду использовать фреймворк Vue.js. Основными его преимуществами являются:
	\begin{description}
		\item[Реактивность] Данные связываются с интерфейсом, который изменяется автоматически, когда данные обновляются.  
		\item[Легковестность и скорость] по заявлению разработчиков, Vue.js быстрее и меньше других аналогичных фреймворков~\autocite{vue}. 
		\item[Независмость от серверной части] Хотя Vue.js может интегрироваться в сервер для улучшения производительности и работы с поисковыми системами~\autocite{vue2}, однако подключение фреймворка через HTML-тег \textbf{script} успешно работает для ненагруженных страниц.  
		
	\end{description}    
 
\section{Постановка задачи}
Решение задачи будет представлено в виде веб-сервиса на фреймворке Express на Node.js, имеющего следующие отличительные функции и особенности:
 \begin{itemize}
	\item Создание и проведение опросов. 
	\item Динамическое отображение результатов опроса на странице.
	\item Каждый опрос имеет короткие ссылки для голосования и просмотра результатов.
	\item Параллельное проведение нескольких опросов на одном веб-сервисе.
	\item Защита от вредоносного искажения результатов. 
	\item Взаимодействие клиента и сервера через технологию WebSocket.
	\item Открытый исходный код под свободной лицензией.
\end{itemize} 
 
\section{Аспекты реализации}
\subsection{Структура проекта на Node.js}
\label{subsec:project_structuture}
Для создания основы проекта с использованием Express я использовал генератор проектов \textbf{express-generator}~\autocite{npm}. Эта утилита позволяет избавится от необходимости конфигурировать встроенный сервер и параметры фреймворка. После ее выполнения в выбранном каталоге генерируется простое, готовое для запуска веб-приложение структуры, представленной на листинге~\ref{list:express-dir}.
\begin{ListingEnv}
\dirtree{%
.1 sample-app.
.2 app.js.
.2 bin.
.3 www.
.2 package.json.
.2 public.
.3 images.
.3 javascripts.
.3 stylesheets.
.4 style.css.
.2 routes.
.3 index.js.
.3 users.js.
.2 views.
.3 error.pug.
.3 index.pug.
.3 layout.pug.
} 
\caption{Структура шаблонного проекта Express}
\label{list:express-dir}   
\end{ListingEnv}

 Разберем назначение некоторых файлов и каталогов:
 \begin{description}
 	\item[package.json] файл с указаниями для пакетного менеджера npm для Node.js. Содержит список всех пакетов, от которых зависит приложение, и точку входа для Node.js.  
 	\item[bin/www] точка входа исполняемого кода JavaScript. В нем создаются и соединяются объекты HTTP-сервера и Express-приложения. Все остальные файлы кода для Node.js являются JavaScipts-модулями, которые подключаются в \textbf{www}, либо в других модулях, уже подключенных в \textbf{www}.
 	\item[app.js] файл с настройками, касающихся конкретно работы Express. Также в нем подключаются файлы с обработчиками путей запросов.   
 	\item[public/] в настройках проекта этот каталог сконфигурирован как общедоступный для  HTTP-запросов вида GET. В нем хранятся статичные файлы для клиентской части веб-сервиса. Во время его работы эти файлы доступны по запросам вида \textbf{http://project.com/stylesheets/style.css}.
 	\item[routes/]   содержит обработчики путей запросов, поступающих в Express.
 	\item[views/]   содержит файлы шаблоны веб-страниц в формате PUG.  
 \end{description}


Такой шаблон проекта значительно упростил разработку сервиса, потому что любой написанный поверх него функционал сразу можно запустить и протестировать, а готовое файловое устройство принуждает к структурированному стилю написания кода. В конечном проекте структура каталогов приложения остается такой же, и только изменяются и добавляются в них файлы кода.      
\subsection{Использование фреймворка Express}
Разработка веб-сервиса на Express сводится к написанию обработчиков путей HTTP-запросов поступающих на сервер.
\begin{ListingEnv}[H]
\begin{lstlisting}[language=JavaScript]
router.get("/get_polled",function(req, res, next){
 let id = db.create_polling_session({
	title: "poll",
	options: [
	  { title: "option1" },
	  { title: "option2" },
	  { title: "option3" }
 ]});
 res.status(200).json({link:db.polling_sessions[id].view_link});
});
	
\end{lstlisting}
\caption{Пример функции обработчика GET-запроса}
\label{list:routing_example}
\end{ListingEnv}
В примере (листинг~\ref{list:routing_example}) для объекта \textbf{router}, который затем можно экспортировать для использовании в приложении Express, определяется поведение при поступлении GET-запроса по адресу \textbf{sample.com/get\_polled}. Для этого используется анонимная функция обратного вызова, которая исполняется каждый раз при поступлении запроса. Аргументы \textbf{req}, \textbf{res}, \textbf{next} представляют, соотвественно, объекты с данными о запросе(\textbf{REQuest}), с функциями ответа (\textbf{RESponse}) и функцию для вызова следующего подходящего обработчика пути. В примере происходит взаимодействие с моделью данных и отправка клиенту ответа с результатами взаимодействия в формате JSON.

Стоит отметить, что возможность использовать регулярные выражения и сопоставление с образцом для путей запросов, а также выстраивать цепочки из обработчиков позволило выстроить в веб-сервисе относительно сложную логику ответов на запросы. Хотя описание этой логики в исходном коде не занимает много места.  

\subsection{Использование WebSocket совместно с фреймворком Express}
Для того чтобы добавить функциональность WebSocket в приложение, я воспользовался пакетом \textbf{express-ws}~\autocite{express-ws}. Этот пакет добавляет в Express функцию для создания обработчиков путей запроса соединения по протоколу WebSocket. Чтобы добавить желаемый функционал код из пакета модифицирует прототипы класса Router внутри Express и добавляет в указанный веб-сервер обработчик запросов типа:
\begin{lstlisting}[language]
GET /api/websocket HTTP/1.1
Host: server.example.com
Upgrade: websocket
Connection: Upgrade
...
\end{lstlisting}
После этого в контексте фреймворка можно писать обработчики путей для WebSocket:
\begin{ListingEnv}[H]
	\begin{lstlisting}[language=JavaScript]
app.ws('/echo', function(ws, req) {
	ws.on('message', function(msg) {
		ws.send(msg);
	});
});
	\end{lstlisting}
	\caption{Пример функции обработчика WS-запроса}
	\label{list:ws_example}
\end{ListingEnv}
В примере (листинг~\ref{list:ws_example}), в отличие от примера в листинге~\ref{list:routing_example}, нет объекта ответа \textbf{res}, но есть объект \textbf{ws} --- асинхронно открывающийся веб-сокет, представляющий соединение между конкретным клиентом и сервером. К этому объекту уже можно привязать функции обработчики для событий изменения состояния вебсокета или получения данных. Когда состояние вебсокета изменится на \textbf{OPEN}, для объекта \textbf{ws} можно вызвать метод \textbf{send} и отправить клиенту любые данные.   

Как было сказано в подразделе~\ref{subsec:project_structuture} в шаблоне \textbf{express-generator} создание объекта HTTP-сервера и связывание его с экземпляром объекта Express происходит в файле \textbf{bin/www}, который является входной точкой исполнения кода для Node.js. Такое устройство приложения по-умолчанию не подходит для \textbf{express-ws}, которому нужно вмешаться в устройство сервера и экземпляра Express. Чтобы это исправить, необходимо внести определенные изменения в файлы \textbf{app.js} и \textbf{bin/www}.

Пользуясь тем, что \textbf{app.js} импортируется в начале \textbf{www/bin}, а значит и интерпретируется раньше чем его основной код, я перенес создание экземпляров HTTP-сервера и Express в \textbf{app.js}. Там же происходит внедрение \textbf{express-ws}.
\begin{ListingEnv}[H]
\begin{lstlisting}[language=JavaScript]
	...
var express = require('express');
var http = require('http');
	...
var app = express();
app.server = http.createServer(app); 
var expressWS = require('express-ws')(app, app.server);
	...
module.exports = app;
\end{lstlisting}
\caption{Изменения в app.js}
\label{list:appjs-ws}
\end{ListingEnv}  
Внутри \textbf{bin/www} создание сервера я заменил на обращение к ссылке на веб-сервер из экземпляра Express.
\begin{ListingEnv}[H]
\begin{lstlisting}[language=JavaScript]
var app = require('../app');
	...
var server = app.server;
server.listen(port);
server.on('error', onError);
server.on('listening', onListening);
	...
\end{lstlisting}
\caption{Изменения в bin/www}
\label{list:www-ws}
\end{ListingEnv}
Таким образом, можно воспользоваться технологией WebSocket, не нарушая структуру шаблонного приложения на фреймворке Express. 

\subsection{Проектирование и разработка модели данных}
Модель данных веб-сервиса представляет информацию о его текущем состоянии и о пользовательских данных. Обычно часть информации хранится в базе данных, отделенной от основного сервера. Такой подход удобен для масштабных проектов, где важна сохранность данных в случае неполадок или ошибок в сервисе. Однако для данного проекта интегрирование работы с базой данных будет являться только обременением, потому что в нем нет важной информации, которую нужно хранить между перезапусками.

В этом проекте я использовал JavaScript-объект, который содержит все другие объекты-представления данных и методы взаимодействия с ними. Этот объект и весь сопутствующий код описываются в отдельном модуле \textbf{database.js}, который экспортирует экземпляр объекта. Таким образом, код в модуле интерпретируется только один раз, и, в какой бы точке проекта не был бы импортирован объект с моделью данных, это всегда будет один и тот же экземпляр.

Прежде всего нужно было определится с тем, какие данные необходимо хранить модели и как с ними будет взаимодействовать остальная логика сервиса. Для этого был составлен список требований, которым должна отвечать модель:
 
\begin{itemize}
 \item В модели может существовать неограниченное количество параллельных сессий, которые могут перемещаться между своими опросами.
 \item Каждая сессия имеет две короткие ссылки для просмотра и участия в опросе. 
 \item Имея короткую ссылку, код должен уметь быстро переходить к данным о сессии, которой она принадлежит.
 \item Код должен быстро получать список пользователей, показывающих опрос или в нем участвующих.
 \item Пользователь может голосовать и перезагружать страницу неограниченное число раз, не вызывая подтасовку результатов.
 \item Модель должна быть устойчивой к добавлению новых видов взаимодействия пользователей с сервисом.
\end{itemize}
 
На основе этих требований я разработал следующую модель представления данных (см. листинг~\ref{list:model}). В основе реализации модели лежит использование некоторых JavaScript-объектов как словарей, где ключи это свойства объекта, например словарь сессий опроса \textbf{polling\_sessions} или словарь всех пользователей подключенных к сессии для просмотра опроса \textbf{views}. Ключи генерируются, как уникальные случайные строки. Стоить отметить, что для таких объектов нельзя определять методы, потому что в JavaScript метод также является свойством, и тогда возникнут трудности, если понадобится перебрать все элементы словаря. 
 
\begin{ListingEnv}
\begin{lstlisting}[language=JavaScript]
{
 polling_sessions: {
  aebd99ccd3: {
	views: { "6feb3f116c": {WebSocket Object} },
	slaves: { "641b151c81": {WebSocket Object} },
	password: "admin",
	view_link:"ecdd",
	slave_link:"79dd4b",
	state: {
		current_poll: 0,
		title: "poll1",
		options: [
		 { title: "option1", count: 1 },
		 { title: "option2", count: 0 },
		 { title: "option3", count: 0 }
		],
		voters: {"641b151c81": 0 }
	 },
	polls: [
		{title: "poll1",
		 options: [
		 	{ title: "option1" },
			{ title: "option2" },
			{ title: "option3" }
		 ]
		},
		{title: "poll2",
		 options: [
			{ title: "option3" },
			{ title: "option4" },
			{ title: "option5" }
		 ]
		}
	]
  }
 },
 short_links: {
	"79dd4b":{ type:"slave", session: {Session Object}},
	"ecdd":	 { type: "view", session: {Session Object}}
	}
};
\end{lstlisting}
\caption{Пример объекта модели во время работы приложения}
\label{list:model}
\end{ListingEnv}
 
 Каждый пользователей в сессии представлен лишь уникальным идентификатором, по которому доступен экземпляр WebSocket. Существует два вида пользователей: просматривающие результаты опросов и в нем участвующие. Для краткости такие пользователи будут называться \textbf{view} и \textbf{slave} соотвественно. Если пользователь перезагрузит страницу, то его экземпляр WebSocket изменится, но по идентификатору модель его распознает и обновит экземпляр. Также по идентификаторам модель отслеживает, кто и за что проголосовал в опросе. Отдать два голоса невозможно в принципе. 
 
 Для объектов, которые не являются словарями, например сессии опросов \textbf{<<aebd99ccd3>>}, можно определять любые методы, инкапсулирующие взаимодействие данных и пользователей. Так, например, в файле \textbf{database.js} функция \textbf{locaс\_db.patсh\_polling\_session} добавляет к передаваемому ей экземпляру новой сессии все методы, которые меняют ее состояние. 
 
\subsection{Взаимодействие клиента и сервера через протокол WebSocket}
В клиентской части веб-приложения за работу с WebSocket отвечает статичный JavaScript-файл \textbf{ws.js}, единый как для \textbf{slave}, так и для \textbf{view} пользователей. Этот скрипт извлекает идентификатор пользователя из страницы или cookie, если он там есть, а затем открывает WebSocket-соединение по адресу:     
 \begin{Verb}
 	ws:[URL веб-сервиса]/ws/[короткая ссылка]  
 \end{Verb}

Короткая ссылка в адресе соединения позволяет серверу понять, к какой сессии принадлежит пользователь. Сразу после открытия подключения (событие \textbf{open} у объекта WebSocket), клиент отправляет свой идентификатор серверу и переходит в режим ожидания ответных сообщений, содержащих, например, состояние страницы.

\begin{ListingEnv}[H]
	\begin{lstlisting}[language=JavaScript]
var db = require('../database.js');
	...
router.ws('/ws/:shortlink',function(ws,req){
 let link = db.short_links[req.params.shortlink];
 if(!link)
  next();
 let id_type = link.type;
 let poll_session = link.session;
 let got_id = false;
 ws.on('message', function(data){
  got_id = true;
  data = JSON.parse(data);
  if(!poll_session.verify_update(data.session_id,id_type,ws))
   ws.close();
 });
 setTimeout(function(){
  if(!got_id)
   ws.close();
 },4000);
});
\end{lstlisting}
\caption{Обработчик нового WebSocket соединения, поступившего на сервер}
\label{list:poll-ws}
\end{ListingEnv}

На стороне сервера новое WebSocket соединение не сразу передается под управление модели. Как видно в листинге~\ref{list:poll-ws}, новое соединение должно пройти ряд проверок перед отправкой в модель:
   \begin{itemize}
   	\item Короткая ссылка из пути запроса должна существовать в модели.
   	\item В течение четырех секунд клиент должен прислать приемлемый JSON со своим идентификатором, иначе соединение будет закрытою.
   \end{itemize}
и в самой модели внутри функции \textbf{verify\_update}:
    \begin{itemize}
 	\item Идентификатор пользователя должен существовать в сессии.
 	\item Заявленный через короткую ссылку тип пользователя должен совпадать с типом полученного идентификатора.
 	\end{itemize}
Провал хотя бы одной проверки приводит к закрытию соединения и освобождению идентификатора. Если же проверки успешны, то объект сохраняется в сессию, и функция  \textbf{polling\_session\_object.patch\_websocket} расширяет обработчик события сообщения обработчиками \textbf{polling\_session\_object.message\_from\_slave} или \textbf{polling\_session\_object.message\_from\_view}, которые получают через аргументы не только новые данные, но идентификатор и ссылку на объект WebSocket.

Как показало тестирование и отладка приложения, оказалось очень важно удалять ненужные обработчики событий, так, например, неубранный обработчик на событие \textbf{message}, приводил к множественным вызовам  \textbf{polling\_session\_object.patch\_websocket}, которые добавляли еще больше обработчиков на один WebSocket. 

На основе этой архитектуры я выстроил простой протокол сообщений в формате JSON (см. листинг~\ref{list:message-ws}), через которые клиенты могут получать новые состояния страниц, голосовать на опросах, повыситься до владельца опроса, листать опросы и так далее.
 \begin{ListingEnv}
 	\begin{lstlisting}[language=JavaScript]
{
 "session_id": "4e643652f903975fbdd304e0115bfcb6f2a53f35",
 "type": "new_state",
 "state": {
   "title": "poll",
   "options": [
	{ "title": "option1", "count": 3 },
	{ "title": "option2", "count": 0 },
	{ "title": "option3", "count": 10 },
   ]
  }
}
 	\end{lstlisting}
 	\caption{Пример сообщения клиенту от сервера с текущим состоянием приложения}
 	\label{list:message-ws}
 \end{ListingEnv}

В случае изменения данных в модели, например \textbf{slave}-клиент проголосовал в опросе, нужно послать всем пользователям обновленные данные. Если просто вызывать функцию отправки данных всем \textbf{view}-клиентам в конце callback-функции обработки сообщения от одного \textbf{slave}-клиента, то даже при паре пользователей, непрерывно меняющих голос в опросе, сервер перестанет справляться с объемами отправляемой информации, и пользователям будет приходить запоздалая и неактуальная информация.

Чтобы решить эту проблему, я воспользовался пакетом debounce~\autocite{debounce} для Node.js. Debounce предоставляет обертку для функций, не позволяющую вызывать функцию чаще, чем в определенный промежуток времени. Вызовы, произошедшие в промежуток, откладываются в очередь и выполняются через этот промежуток.
 \begin{ListingEnv}[H]
	\begin{lstlisting}[language=JavaScript]
polling_session_object.send_state_all = debounce(function() {
 var state = { title: polling_session_object.state.title, options: polling_session_object.state.options };
 for (var id in polling_session_object.views) {
		...
 view.send(JSON.stringify({ session_id: id, type: "new_state", state: state }));
  }
 for (var id in polling_session_object.slaves) {
		...
 view.send(JSON.stringify({ session_id: id, type: "new_state", state: state }));
 }
 this.send_state_all.clear();
},200);
\end{lstlisting}
\caption{Использование debounce для контроля отправки нового состояния всем пользователям}
\label{list:debounce-exmpl}
\end{ListingEnv}      

Также важно очищать очередь вызовов после отработки кода через \textbf{debounce.clear} как в примере на листинге~\ref{list:debounce-exmpl}. Тогда нескольким вызовам обернутой функции, произошедшим в один промежуток, соотвествует только одно исполнение кода.

\subsection{Создание пользовательского интерфейса}
В данном веб-сервисе существует два основных вида страниц, которые я реализовал в первую очередь. Страница просмотра результатов опроса и страница голосования, которые в модели данных представляются \textbf{view} и \textbf{slave} пользователями соответственно.

Страница просмотра результатов не только динамично обновляет результаты опроса, но также показывает ссылку по которой можно присоединится к нему и, при вводе секретной фразы, указанной при создании опроса, дает пользователю контроль над проведением опроса.

Для того что бы отправлять страницы пользователям, я использую шаблонный движок Pug. Его синтаксис состоит из смеси JavaScript и видоизмененного HTML (см. листинг~\ref{list:pug-lay}). C помощью табуляции описывается древовидная структура страницы, точка обозначает блок сплошного текста, а \textbf{<<\#{...}>>} окружает JavaScript-код, который интерпретируется на стороне сервера, а результат записывает на его место~\autocite{pug}. Pug --- мощный инструмент шаблонизации, но в данном проекте я использую его, чтобы передать пользователю только основу страницы, в которой указаны ссылки на загрузку частей фреймворка и начальные данные о состоянии страницы.
 \begin{ListingEnv}
	\begin{lstlisting}[language=pug]
doctype html
html
 head
  meta(charset="utf-8")
  script(src="/javascripts/js.cookie.js")
  link(rel="stylesheet" href="https://unpkg.com/vue-material@beta/dist/vue-material.min.css")    
 body
  script(type='text/javascript').
  \end{lstlisting}
  \begin{lstlisting}[language=JavaScript]
     var sessionInfo = {
	 	id : "#{session_id}",
	 	type : "#{session_type}",
		viewLink: "#{view_shortlink}",
		slaveLink:  "#{slave_shortlink}"
	}  
\end{lstlisting}
\begin{lstlisting}[language=pug]
  main#app 
	block content
  script(src="/javascripts/ws.js")
  script(src="https://cdn.jsdelivr.net/npm/vue/dist/vue.js")
  script(src="https://unpkg.com/vue-material@beta")
  script.
	Vue.use(VueMaterial.default);
  block footer_scripts
	\end{lstlisting}
	\caption{Шаблон разметки страницы на языке Pug}
	\label{list:pug-lay}
\end{ListingEnv}      

Основную работу по отображению элементов выполняет фреймворк Vue.js. Ядро фреймворка загружается на страницу из CDN (пер. Сеть доставки содержимого) через тег \textbf{script}. После этого в последующих скриптах можно описывать компоненты страницы как объекты Vue. 

В файлах \textbf{javascripts/slave\_ui.js} и \textbf{javascripts/view\_ui.js} я определил Vue-компоненты для соответствующих страниц. Компоненты напрямую связаны с данными, например, о состоянии страницы, и если изменяются данные, то компоненты перерисовываются. Синтаксис шаблонов Vue позволяет описывать элементы через условия и циклы, чем я и пользуюсь, чтобы отображать опросы с любым количеством вариантов.

В файле \textbf{javascripts/ws.js} описана логика соединения с сервером через WebSocket, но ее запуск обернут в метод  
\textbf{sessionInfo.init\_WS}. Это связано с тем, что экземпляру Vue требуется время, чтобы полностью создаться и связать данные с компонентами, поэтому вызов \textbf{init\_WS} происходит только после события \textbf{created} в экземпляре Vue. Тогда фреймворк уже отслеживает изменения данных.     

\section{Обзор проекта}
\subsection{Внутреннее устройство}
За время разработки общая структура проекта не менялась кардинально, JavaScript-модули и другие файлы кода, реализующий новый функционал, добавлялись в соответствующие каталоги. В текущей итерации разработки проект имеет структуру, представленную на листинге~\ref{list:poll-dir}. Некоторые из новых файлов уже упоминались выше, но все же разберем каждый из них:

\begin{ListingEnv}[p]
\dirtree{%
	.1 poll\_app.
	.2 bin.
	.3 www.
	.2 public.
	.3 images.
	.3 javascripts.
	.4 builder\_ui.js.
	.4 index\_ui.js.
	.4 slave\_ui.js.
	.4 view\_ui.js.
	.4 js.cookie.js.
	.4 ws.js.
	.3 stylesheets.
	.4 builder\_style.css.
	.4 index\_style.css.
	.4 slave\_style.css.
	.4 view\_style.css.
	.2 routes.
	.3 poll.js.
	.2 views.
	.3 error.pug.
	.3 index.pug.
	.3 layout.pug.
	.3 builder.pug.
	.3 view.pug.
	.3 slave.pug.
	.2 app.js.
	.2 database.js.
	.2 package.json.
	.2 LICENSE.
	.2 README.md.
}
\caption{Структура проекта Poll\_app}
\label{list:poll-dir}
\end{ListingEnv}
\begin{description}
	\item[javascripts/\dots\_ui.js] Это JavaScript-файлы, в которых содержится описание интерфейса каждой страницы в контексте фреймворка Vue.js. 
	\item[javascripts/ws.js] В этом файле описана логика соединения клиента и сервера по протоколу WebSocket.
	\item[javascripts/js.cookie.js] Свободная легковесная библиотека для работы с куки на стороне клиента.
	\item[stylesheets/\dots\_style.css] Каскадные таблицы стилей для каждой страницы.
	\item[views/ \dots\space.pug] Файлы HTML-шаблонов страниц в формате Pug. Каждый шаблон подключает файл \textbf{layout.pug}, в котором описаны общие для всех страниц HTML-элементы. 
	\item[routes/poll.js] Файл, в котором описаны все пути HTTP-запросов, поступающих в приложение, и их обработчики в контексте фреймворка Express.
	\item[database.js] JavaScript-модуль, экспортирующий экземпляр объекта модели данных в приложении.  
\end{description}

Приложение состоит из четырех страниц. По адресам <<\textbf{/}>> и <<\textbf{/index}>>, доступна начальная страница. На ней отображаются все текущие и доступные сессии опросов. К текущим сессиям можно присоединиться как зритель или участник. А доступные опросы можно начать, если ввести кодовое слово, с которым опрос был создан.

На странице <<\textbf{/build\_poll}>> можно создать опрос и отправить его на сервер. Указанное при создании кодовое слово затем позволит запустить опрос или управлять им во время показа.

Страницы просмотра и участия в опросе имеют уникальные для каждого опроса короткие ссылки. Страница участия оптимизирована в первую очередь для мобильных устройств. Со страницы просмотра можно управлять ходом опроса, если ввести кодовое слово. Содержание этих страниц обновляется динамически, когда изменяется состояние опроса.     
          
\subsection{Установка и использование проекта}  
Исходные коды проекта можно получить в Git репозитории по ссылке:
\href{https://github.com/Jeday/Sycidium.git}{github.com/Jeday/Sycidium.git}

Для GNU/Linux систем:
\begin{enumerate}
	\item Установите последние версии Node.js и npm.
	\item Склонируйте файлы приложения в отдельную папку.
	\item В корне приложения выполните \textbf{npm install}, чтобы установить зависимости.
	\item Запустите приложение через \textbf{npm start}.
	\item Приложение доступно по адресу \textbf{http://localhost:3000/}
\end{enumerate}

Чтобы запусть приложение в рабочем режиме с произвольным портом нужно изменить порт в файле \textbf{package.json} и выполнить \textbf{npm run prod}.
  \begin{ListingEnv}
 \begin{lstlisting}[language=JavaScript]
{
 "name": "poll-app",
 "version": "0.7.0",
 "private": true,
 "scripts": {
 	"start": "node ./bin/www",
 	"prod": "NODE_ENV=production PORT=80 node ./bin/www"
    },
 "dependencies": {
	...
 }
}

\end{lstlisting}
	\caption{Файл package.json с установленным портом 80}
\label{list:pack-json}
\end{ListingEnv}      



\newpage
\Conc
В рамках данной работы был реализован полноценный веб-сервис для проведения опросов во время публичных выступлений. Веб-сервис позволяет создавать, запускать и управлять опросами.Пользователи могут сразу же поучаствовать в опросе со своего мобильного устройства, перейдя по короткой ссылке, указанной на экране опроса. А их выбор моментально отобразится на главном экране. Также реализована основная защита от вредоносного вмешательства в процесс опроса, не позволяющая испортить результаты или остановить ход опроса. Веб-сервис имеет открытые исходные коды и доступен для развертывания, использования и модификации по свободной лицензии MIT.    

Наиболее очевидным способом улучшения сервиса является добавление больших возможностей для взаимодействия с аудиторией и интеграцией с потоком презентации. К таким функциям можно отнести: чат, отображающийся на экране; слияние опросов и слайдом презентации; разные формы и виды опросов.


\newpage
% Печать списка литературы (библиографии)
\printbibliography[%{}
    heading=bibintoc%
    %,title=Библиография % если хочется это слово
]
% Файл со списком литературы: biblio.bib
% Подробно по оформлению библиографии:
% см. документацию к пакету biblatex-gost
% http://ctan.mirrorcatalogs.com/macros/latex/exptl/biblatex-contrib/biblatex-gost/doc/biblatex-gost.pdf
% и огромное количество примеров там же:
% http://mirror.macomnet.net/pub/CTAN/macros/latex/contrib/biblatex-contrib/biblatex-gost/doc/biblatex-gost-examples.pdf

\appendix
\ifthenelse{\value{worktype} > 1}{%
  \addtocontents{toc}{%
      \protect\renewcommand{\protect\cftchappresnum}{\appendixname\space}%
      \protect\addtolength{\protect\cftchapnumwidth}{\widthof{\appendixname\space{}} - \widthof{Глава }}%
  }%
}{
  \addtocontents{toc}{%
      \protect\renewcommand{\protect\cftsecpresnum}{\appendixname\space}%
      \protect\addtolength{\protect\cftsecnumwidth}{\widthof{\appendixname\space{}}}%
  }%
}


\end{document}
