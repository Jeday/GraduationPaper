% В этом файле следует писать текст работы, разбивая его на
% разделы (section), подразделы (subsection) и, если нужно,
% главы (chapter).

% Предварительно следует указать необходимую информацию
% в файле SETUP.tex

\input{preamble.tex}


\NewBibliographyString{langjapanese}
\NewBibliographyString{fromjapanese}

\begin{document}

\Intro
Технологии проведения публичных выступлений и презентаций затрагивают навыки ораторства и внешний вид, дизайн медиа-сопровождения. Методы взаимодействия с аудиторией традиционно включают в первую составляющую. Выступающий, желающий взаимодействовать со слушающими, должен уже обладать определенным опытом в работе с ними и ограничен устными средствами.Крайне редко возможно почти полностью вовлечь аудиторию в выступление, ведь лишь немногие слушатели готовы,например, задать вопрос или ответить выступающему.


Распространение телефонов и мобильного доступа в интернет, позволяет использовать эти устройства как средства взаимодействия с аудиторией. Проекты, использующие эту идею, реализовывались неоднократно, но ни один из них не закрепился как широко используемый в публичных выступлениях. В первую очередь, идея взаимодействия с публикой через телефоны реализовывалась под конкретные единичные выступления. Последующие реализации, хотя и обладают обширным функционалом, в виде  опросов, голосований и чатов, представлют собой отдельные веб-сервисы, направленные на монетизацию с пользователей. Все проекты закрыты проприетарными лицензиями и требуют от пользователей загрузки презентации на сторонний сервер.

Данная работа посвящена разработке проекта портативного веб-сервиса под свободной лицензией, который позволит проводить опросы аудитории во время публичных выступлений без привлечения сторонних сервисов. Свободная лицензия позволит любому человеку изменять и расширять возможности сервиса под свои нужды.

Задача по созданию такого проекта включает разработку как веб-интерфейса пользователя (фронтенд), так и  внутренней логики сервиса(бэкенд), которые в совокупности обеспечат динамичное отображение результатов опросов.
   
\section{Исследование предметной области}
\subsection{Обзор существующих решений}
Как и упоминалось ранее, для опросов аудитории уже существует немалое число инструментов, однако в основой массе это закрытые решения в виде веб-сервисов:
\begin{itemize}
	\item polleverywhere.com
	\item directpoll.com
	\item sli.do
	\item ficus.io
\end{itemize}
На этих сайтах и других подобных можно бесплатно в первый раз провести опрос или даже презентацию, но повторные показы и дополнительные функции ограничены для пользователей, не оплативших услуги сайтов. Более того, даже оплативший пользователь ограничен средствами и функциями сайта и не может модифицировать или изменить инструмент под свои нужды и цели.

Также стоит упомянуть об инструментах опросов, не использующих только Интернет(http://www.ombea.com/). Такие решения применяются в университетах США(http://www.nea.org/home/34690.html) и отличаются низкой способностью к масштабированию и высокой ценной как системы, так и индивидуальных приборов голосования.

\subsection{Обзор инструментов разработки}
При создании веб-сервиса самую важную роль занимает разработка серверной части. Так как веб-сайт должен динамически взаимодействовать с сервером, то архаичная связка из веб-сервера и FastCGI/CGI приложения очевидно не подойдет. Для решения данной задачи необходимо выбрать один из множества современных веб-фреймворков (https://ru.wikipedia.org/wiki/Сравнение\_каркасов\_веб-приложений), как основу для проекта. Отметим основные необходимые для задачи черты фреймворков:
 \begin{itemize}
 	\item легковесность
 	\item инкапсуляция веб-сервера
 	\item наличие актуального функционала(JSON,AJAX,websocket)
 \end{itemize}
Рассмотрим несколько популярных фреймворков:
 \begin{itemize}
 	\item Django --- фреймворк на языке Python. Хотя на нем можно реализовать необходимый нам функционал, но его врядли можно назвать легковесным. Django в первую очередь предназначен для создания больших многостраничных сайтов и сервисов, которые будет длительное время поддерживать команда разработчиков и администраторов. Наличие бесполезного для задачи функционалла негативно сказывается на времени освоения и разработки. (https://www.djangoproject.com/)  
 	\item Ruby on Rails --- фреймворк на языке Ruby. Основными минусами Ruby on Rails являются проксирование через отдельный веб-сервер и общая сложность освоения как фреймворка, так и самого языка.Стоит также отметить, что этот фреймворк сильно опирается на архитектуру модель-представление-контроллер, реализация которой усложняет задачу для небольшого приложения.  (https://rubyonrails.org/)  
 	\item Express --- фреймворк на языке Javasciprt, запускаемый на платоформе Node.js. Express  инкапсулирует веб-сервер,представляя только абстракцию в виде объектов HTTP запроса и ответа,а необходимый для приложения функционал, например WebSocket и шаблонизация , добавляются через совместимы модули Node.js. Так же фреймворк не затрагивает клиентскую часть веб-приложения.(https://expressjs.com/ru/) Express своевременно обновляется, имеет обширную документацию и является популярным выбором среди разработчиков из-за своей простоты и понятности. Основным опасением является производительность однопоточной архитектуры Node.js, однако для небольших и средних приложений Node.js и Express показывают удовлетворительные результаты(https://www.researchgate.net/publication/286594024\_Performance\_Comparison\_and\_Evaluation\_of\_Web\_Development\_Technologies\_in\_PHP\_Python\_and\_Nodejs) 
 \end{itemize}
Таким образом, для решения рассматриваемой задачи фреймворк Express подходит идеально. Так как он не затрагивает клиентскую часть сервиса, то для нее можно использовать любой легковесный JavaScript фреймворк для создания пользовательского интерфейса. 
\section{Обзор Проекта}
Решение поставленной задачи будет представлено в виде веб-сервиса на фреймворке Express на Node.js, имеющего следующие отличительные функции и особенности:
 \begin{itemize}
	\item Создание и проведение опросов. 
	\item Динамическое отображение результатов опроса на странице.
	\item Каждый опрос имеет короткие ссылки для голосования и просмотра результатов.
	\item Параллельное проведение нескольких опросов на одном веб-сервисе.
	\item Защита от вредоносного искажения результатов. 
	\item Взаимодействие клиента и сервера через технологию WebSocket.
	\item Открытый исходный код под свободной лицензией.
\end{itemize} 
 
 
 
 
 
 
 
 
 
 
 
 
 
   
% Если typeOfWork в SETUP.tex задан как 2 или 3, то начинать
% надо не с section (раздел), а с главы (chapter)
\section{Несколько примеров в~\LaTeX{}}
\label{sec:examples}

Некоторые часто используемые
команды приведены в качестве примера ниже (и варианты — в
комментариях). Мы рекомендуем внимательно прочесть данный
текст и изучить его исходный код прежде, чем начинать писать
свой собственный. Кроме того, можно дать и такой совет: идущий
ниже текст не убирать до самого конца, а просто оставлять его
позади своего собственного текста, чтобы в любой момент можно
было проконсультироваться с данными примерами.

\subsection{Как вставлять листинги и рисунки}

Для крупных листингов есть два способа. Первый красивый, но в нём не допускается
кириллица (у вас может встречаться в комментариях и
печатаемых сообщениях), он представлен на листинге~\ref{list:hwbeauty}.
\begin{ListingEnv}[H]% буква H означает Here, ставим здесь,
% элементы, которые нежелательно разрывать обычно не ставят
% посреди страницы: вместо H используется t (top, сверху страницы),
% или b (bottom) или p (page, на отдельной странице)
\begin{lstlisting}
#include <iostream>
using namespace std;

int main()
{
    cout << "Hello, world" << endl;
    system("pause");
    return 0;
}
\end{lstlisting}
%следующую команду для генерации подписи можно опустить,
% хотя рекомендуется все специальные элементы (таблицы, рисунки,
% листинги) подписывать. Если подпись пропустить, листинг также не получит
% номера и на него не сошлёшься в будущем
\caption{Программа “Hello, world” на \protect\cpp}
% далее метка для ссылки:
\label{list:hwbeauty}
\end{ListingEnv}

Второй не такой красивый, но без ограничений (см.~листинг~\ref{list:hwplain}).
\begin{ListingEnv}[H]
\begin{Verb}

#include <iostream>
using namespace std;

int main()
{
    cout << "Привет, мир" << endl;
}
\end{Verb}
\caption{Программа “Hello, world” без подсветки}
\label{list:hwplain}
\end{ListingEnv}

Можно использовать первый для вставки небольших фрагментов
внутри текста, а второй для вставки полного
кода в приложении, если таковое имеется.

Если нужно вставить совсем короткий пример кода (одна или две строки), то выделение  линейками и нумерация может смотреться чересчур громоздко. В таких случаях можно использовать окружения \texttt{lstlisting} или \texttt{Verb} без \texttt{ListingEnv}. Приведём такой пример с указанием языка программирования, отличного от заданного по умолчанию:
\begin{lstlisting}[language=Haskell]
fibs = 0 : 1 : zipWith (+) fibs (tail fibs)
\end{lstlisting}
Такое решение~--- со вставкой нумерованных листингов покрупнее
и вставок без выделения для маленьких фрагментов~--- выбрано,
например, в книге Эндрю Таненбаума и Тодда Остина по архитектуре
компьютера~\autocite{TanAus2013} (см.~рис.~\ref{fig:tan-aus}).

Наконец, для оформления идентификаторов внутри строк
(функция \lstinline{main} и тому подобное) используется
\texttt{lstinline} или, самое простое, моноширинный текст
(\texttt{\textbackslash texttt}).

\begin{figure}[p]% p означает, что нужно выделить для рисунка
% отдельную страницу; применяется для больших рисунков
\centering
%Здесь могла быть ваша лягушка.
\includegraphics[width=\textwidth]{img/tan-aus.png}
\caption{\label{fig:tan-aus}Пример оформления листингов в~\autocite{TanAus2013}}
\end{figure}

Использовать внешние файлы (например, рисунки) можно и на \href{http://overleaf.com}{overleaf.com}: ищите кнопочку upload.

\subsection{Как оформить таблицу}

Для таблиц обычно используются окружения table и tabular --- см. таблицу~\ref{tab:widgets}. Внутри окружения tabular используются специальные команды пакета booktabs — они очень красивые; самое главное: использование вертикальных линеек считается моветоном.

\begin{table}
\centering
\caption{\label{tab:widgets}Подпись к таблице --- сверху}
\begin{tabular}{llr}
\toprule
\multicolumn{2}{c}{Item} \\
\cmidrule(r){1-2}
Животное  & Описание    & Цена (\$) \\
\midrule
Gnat      & per gram    & 13.65      \\
          & each        & 0.01       \\
Gnu       & stuffed     & 92.50      \\
Emu       & stuffed     & 33.33      \\
Armadillo & frozen      & 8.99       \\
\bottomrule
\end{tabular}
\end{table}

\subsection{Как набирать формулы}

\LaTeX{} is great at typesetting mathematics. Let $X_1, X_2, \ldots, X_n$ be a sequence of independent and identically distributed random variables with $\text{E}[X_i] = \mu$ and $\text{Var}[X_i] = \sigma^2 < \infty$, and let
$$S_n = \frac{X_1 + X_2 + \cdots + X_n}{n}
      = \frac{1}{n}\sum_{i}^{n} X_i$$
denote their mean. Then as $n$ approaches infinity, the random variables $\sqrt{n}(S_n - \mu)$ converge in distribution to a normal $\mathcal{N}(0, \sigma^2)$.

\subsection{Как оформлять списки}

Нумерованные списки (окружение enumerate, команды item)…

\begin{enumerate}
  \item Like this,
  \item and like this.
\end{enumerate}

\dots маркированные списки \dots

\begin{itemize}
  \item Like this,
  \item and like this.
\end{itemize}

\dots списки-описания \dots

\begin{description}
  \item[Word] Definition
  \item[Concept] Explanation
  \item[Idea] Text
\end{description}

\Conc

Помните, что на все пункты списка литературы должны быть ссылки. \LaTeX\ просто не добавит информацию об издании из bib"/файла, если на это издание нет ссылки в тексте. Часто студенты используют в работе  электронные ресурсы: в этом нет ничего зазорного при одном условии: при каждом заимствовании следует ставить соответствующую ссылку. В качестве примера приведём ссылку на сайт нашего института~\autocite{mmcs}.

Для дальнейшего изучения \LaTeX\ рекомендуем книгу Львовского~\autocite{Lvo2003}: она хорошо написана, хотя и несколько устарела.
Обычно стоит искать подсказки на
\href{http://tex.stackexchange.com/}{tex.stackexchange.com}, а также
читать документацию по установленным пакетам с помощью
команды
\begin{Verb}
texdoc имя_пакета
\end{Verb}
или на \href{http://ctan.org/}{ctan.org}.

% Печать списка литературы (библиографии)
\printbibliography[%{}
    heading=bibintoc%
    %,title=Библиография % если хочется это слово
]
% Файл со списком литературы: biblio.bib
% Подробно по оформлению библиографии:
% см. документацию к пакету biblatex-gost
% http://ctan.mirrorcatalogs.com/macros/latex/exptl/biblatex-contrib/biblatex-gost/doc/biblatex-gost.pdf
% и огромное количество примеров там же:
% http://mirror.macomnet.net/pub/CTAN/macros/latex/contrib/biblatex-contrib/biblatex-gost/doc/biblatex-gost-examples.pdf

\appendix
\ifthenelse{\value{worktype} > 1}{%
  \addtocontents{toc}{%
      \protect\renewcommand{\protect\cftchappresnum}{\appendixname\space}%
      \protect\addtolength{\protect\cftchapnumwidth}{\widthof{\appendixname\space{}} - \widthof{Глава }}%
  }%
}{
  \addtocontents{toc}{%
      \protect\renewcommand{\protect\cftsecpresnum}{\appendixname\space}%
      \protect\addtolength{\protect\cftsecnumwidth}{\widthof{\appendixname\space{}}}%
  }%
}

\section{Пример работы программы}

Здесь длинный листинг с примером работы.

\end{document}
