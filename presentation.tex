%%%%%%%%%%%%%%%%%%%%%%%%%%%%%%%%%%%%%%%%%
% Beamer Presentation
% LaTeX Template
% Version 1.0 (10/11/12)
%
% This template has been downloaded from:
% http://www.LaTeXTemplates.com
%
% License:
% CC BY-NC-SA 3.0 (http://creativecommons.org/licenses/by-nc-sa/3.0/)
%
%%%%%%%%%%%%%%%%%%%%%%%%%%%%%%%%%%%%%%%%%

%----------------------------------------------------------------------------------------
%	PACKAGES AND THEMES
%----------------------------------------------------------------------------------------

\documentclass{beamer}

\mode<presentation> {

% The Beamer class comes with a number of default slide themes
% which change the colors and layouts of slides. Below this is a list
% of all the themes, uncomment each in turn to see what they look like.

%\usetheme{default}
%\usetheme{AnnArbor}
%\usetheme{Antibes}
%\usetheme{Bergen}
%\usetheme{Berkeley}
%\usetheme{Berlin}
%\usetheme{Boadilla}
%\usetheme{CambridgeUS}
%\usetheme{Copenhagen}
%\usetheme{Darmstadt}
%\usetheme{Dresden}
%\usetheme{Frankfurt}
%\usetheme{Goettingen}
%\usetheme{Hannover}
%\usetheme{Ilmenau}
%\usetheme{JuanLesPins}
%\usetheme{Luebeck}
%\usetheme{Madrid}
%\usetheme{Malmoe}
%\usetheme{Marburg}
%\usetheme{Montpellier}
%\usetheme{PaloAlto}
%\usetheme{Pittsburgh}
\usetheme{Rochester}
%\usetheme{Singapore}
%\usetheme{Szeged}
%\usetheme{Warsaw}

% As well as themes, the Beamer class has a number of color themes
% for any slide theme. Uncomment each of these in turn to see how it
% changes the colors of your current slide theme.

%\usecolortheme{albatross}
%\usecolortheme{beaver}
%\usecolortheme{beetle}
%\usecolortheme{crane}
%\usecolortheme{dolphin}
%\usecolortheme{dove}
%\usecolortheme{fly}
%\usecolortheme{lily}
%\usecolortheme{orchid}
%\usecolortheme{rose}
%\usecolortheme{seagull}
\usecolortheme{seahorse}
%\usecolortheme{whale}
%\usecolortheme{wolverine}

%\setbeamertemplate{footline} % To remove the footer line in all slides uncomment this line
\setbeamertemplate{footline}[page number] % To replace the footer line in all slides with a simple slide count uncomment this line

%\setbeamertemplate{navigation symbols}{} % To remove the navigation symbols from the bottom of all slides uncomment this line
}

\usepackage{graphicx} % Allows including images
\usepackage{booktabs} % Allows the use of \toprule, \midrule and \bottomrule in tables
\usepackage [T2A] {fontenc}   % Кириллица в PDF файле
\usepackage [utf8] {inputenc} % Кодировка текста: utf-8
\usepackage [russian] {babel} % Переносы, лигатуры
%----------------------------------------------------------------------------------------
%	TITLE PAGE
%----------------------------------------------------------------------------------------

\title[Выпускная квалификационная работа]{Разработка системы проведения опросов аудитории во время публичных выступлений} % The short title appears at the bottom of every slide, the full title is only on the title page

\author{Е.А.~Тактаров, к.ф.-м.н., доцент Е.\,М.~Андреева} % Your name
\institute[ИММиКН] % Your institution as it will appear on the bottom of every slide, may be shorthand to save space
{
Институт математики, механики и компьютерных наук им. И.И. Воровича  \\ % Your institution for the title page
Южный Федеральный Университет
}
\date{2019} % Date, can be changed to a custom date

\begin{document}

\begin{frame}
\titlepage % Print the title page as the first slide
\end{frame}

\begin{frame}
\frametitle{Содержание} % Table of contents slide, comment this block out to remove it
\tableofcontents % Throughout your presentation, if you choose to use \section{} and \subsection{} commands, these will automatically be printed on this slide as an overview of your presentation
\end{frame}

%----------------------------------------------------------------------------------------
%	PRESENTATION SLIDES
%----------------------------------------------------------------------------------------

\section{Постановка задачи}
\begin{frame}
\frametitle{Постановка задачи}
Создать веб-сервис, отвечающий следующим требованиям:
\begin{itemize}
	\item Функция создания и проведения опросов. 
	\item Динамическое отображение результатов опроса на странице.
	\item Параллельное проведение нескольких опросов на одном развернутом веб-сервисе.
	\item Каждый опрос доступен по коротким ссылкам для голосования и просмотра результатов.
	\item Защита от вредоносного искажения результатов. 
	\item Открытый исходный код под свободной лицензией.
\end{itemize}
\end{frame}

%------------------------------------------------
\section{Обзор инструментов разработки}
\begin{frame}
\frametitle{Обзор инструментов разработки}
\begin{description}
	\item[Node.js] Программная платформа общего назначения для языка JavaScript. 
	\item[Express] Веб-фреймворк Node.js для создания серверной части веб-приложения.
	\item[WebSocket] Протокол связи поверх TCP-соединения, предназначенный для обмена сообщениями между браузером и веб-сервером в режиме реального времени.
	\item[Vue.Js] Веб-фреймворк для создания пользовательского интерфейса в браузерах.  
\end{description}
\end{frame}

%------------------------------------------------
\section{Аспекты Реализации}
%------------------------------------------------

\subsection{Структура веб-приложения}
\begin{frame}
\frametitle{Структура веб-приложения}
\begin{figure}
\includegraphics[width=\linewidth]{img/webdiagram.png}
\end{figure}
\end{frame}

%------------------------------------------------

\subsection{Разработка модели данных}
\begin{frame}
\frametitle{Разработка модели данных}
	\begin{itemize}
		\item Достаточно использовать JavaScript объект, чтобы хранить состояние приложения.
		\item Инкапсулирует всю логику приложения.
		\item Определяется и создается в отдельном модуле \textbf{database.js} и экспортируется из него(всегда один экземпляр)
		\item Внутри рекурсивно списки, Хеш-мапы, объекты со свойствами и методами.
	\end{itemize}
\end{frame}

%------------------------------------------------

\begin{frame}
\frametitle{Разработка модели данных}
	Требования к модели:
\begin{enumerate}
	\item В модели может существовать неограниченное количество параллельных сессий, которые могут перемещаться между своими опросами.
	\item Каждая сессия имеет две короткие ссылки для просмотра и участия в опросе. 
	\item Имея короткую ссылку, код должен уметь быстро переходить к данным о сессии, которой она принадлежит.
	\item Код должен быстро получать список пользователей, показывающих опрос или в нем участвующих.
	\item Пользователь может голосовать и перезагружать страницу неограниченное число раз, не вызывая подтасовку результатов.
	\item Модель должна быть устойчивой к добавлению новых видов взаимодействия пользователей с сервисом.
\end{enumerate}
\end{frame}


%------------------------------------------------

\subsection{Использование WebSocket}
\begin{frame}
\frametitle{Использование WebSocket}
	\begin{enumerate}
		\item Пакет \textbf{express-ws} добавляет в Express обработчики запросов на WebSocket-соединие.
		\item Клиент запрашивает соединение по URL:\\ \textbf{ws:[URL веб-сервиса]/ws/[короткая ссылка]}	  
		\item Данные отправляются в JSON. 
		\item Первое сообщение от клиента всегда содержит его идентификатор и тип сессии. 
		\item Если данные имеют неправильный формат или не совпадают с моделью данных, то сервер закрывает соединение.
	\end{enumerate}
\end{frame}

\begin{frame}
\frametitle{Алгоритм соединения через WebSocket}
\begin{figure}
	\includegraphics[width=\linewidth]{img/wsdiagram.png}
\end{figure}
\end{frame}

\subsection{Debounce}
\begin{frame}
\frametitle{Проблема переполнения запросов}
\begin{figure}
	\includegraphics[width=\linewidth]{img/nodeb.png}
\end{figure}
	Что будет если 10 пользователей проголосуют одновременно?\\
	Каждому пользователю придет 10 сообщений с новым состоянием, и только последнее из них будет актуальным.  
\end{frame}

\begin{frame}
\frametitle{Использование отсроченного вызова}
\begin{description}
	\item[Debounce] пакет для Node.js, предоставляющий обертку для функций, которая откладывает их исполнение на указанный промежуток. 
	\centering
	 \begin{figure}
	 	\includegraphics[width=\linewidth]{img/debounce.png}
	 \end{figure}
\end{description}

\end{frame}





\begin{frame}
\frametitle{Multiple Columns}
\begin{columns}[c] % The "c" option specifies centered vertical alignment while the "t" option is used for top vertical alignment

\column{.45\textwidth} % Left column and width
\textbf{Heading}
\begin{enumerate}
\item Statement
\item Explanation
\item Example
\end{enumerate}

\column{.5\textwidth} % Right column and width
Lorem ipsum dolor sit amet, consectetur adipiscing elit. Integer lectus nisl, ultricies in feugiat rutrum, porttitor sit amet augue. Aliquam ut tortor mauris. Sed volutpat ante purus, quis accumsan dolor.

\end{columns}
\end{frame}

%------------------------------------------------
\section{Second Section}
%------------------------------------------------

\begin{frame}
\frametitle{Table}
\begin{table}
\begin{tabular}{l l l}
\toprule
\textbf{Treatments} & \textbf{Response 1} & \textbf{Response 2}\\
\midrule
Treatment 1 & 0.0003262 & 0.562 \\
Treatment 2 & 0.0015681 & 0.910 \\
Treatment 3 & 0.0009271 & 0.296 \\
\bottomrule
\end{tabular}
\caption{Table caption}
\end{table}
\end{frame}

%------------------------------------------------

\begin{frame}
\frametitle{Theorem}
\begin{theorem}[Mass--energy equivalence]
$E = mc^2$
\end{theorem}
\end{frame}

%------------------------------------------------

\begin{frame}[fragile] % Need to use the fragile option when verbatim is used in the slide
\frametitle{Verbatim}
\begin{example}[Theorem Slide Code]
\begin{verbatim}
\begin{frame}
\frametitle{Theorem}
\begin{theorem}[Mass--energy equivalence]
$E = mc^2$
\end{theorem}
\end{frame}\end{verbatim}
\end{example}
\end{frame}

%------------------------------------------------

\begin{frame}
\frametitle{Figure}
Uncomment the code on this slide to include your own image from the same directory as the template .TeX file.
%\begin{figure}
%\includegraphics[width=0.8\linewidth]{test}
%\end{figure}
\end{frame}

%------------------------------------------------

\begin{frame}[fragile] % Need to use the fragile option when verbatim is used in the slide
\frametitle{Citation}
An example of the \verb|\cite| command to cite within the presentation:\\~

This statement requires citation \cite{p1}.
\end{frame}

%------------------------------------------------

\begin{frame}
\frametitle{References}
\footnotesize{
\begin{thebibliography}{99} % Beamer does not support BibTeX so references must be inserted manually as below
\bibitem[Smith, 2012]{p1} John Smith (2012)
\newblock Title of the publication
\newblock \emph{Journal Name} 12(3), 45 -- 678.
\end{thebibliography}
}
\end{frame}

%------------------------------------------------



\end{document} 